\thispagestyle{empty}
\begin{center}
	\LARGE{\textbf{TRANG THÔNG TIN LUẬN VĂN}}
	\addcontentsline{toc}{chapter}{Trang thông tin luận văn tiếng Việt}
\end{center}

\noindent Tên đề tài luận văn: Sử dụng tính toán song song để lựa chọn đặc trưng dựa trên tiêu chí tỉ lệ vết
\\
Ngành: Cơ sở Toán cho Khoa học Dữ liệu
\\
Mã số ngành: 8480109
\\
Họ tên học viên cao học: Phan Thành Nhân
\\
Khóa đào tạo: 2020 - 2022
\\
Người hướng dẫn khoa học: TS. Nguyễn Thị Thu.
\\
Cơ sở đào tạo: Trường Đại học Khoa học Tự nhiên, ĐHQG.HCM\\
\\
{\bf 1. TÓM TẮT NỘI DUNG LUẬN VĂN:}
\begin{itemize}
	\item[-] Chương 1: Giới thiệu tổng quan về lựa chọn đặc trưng và các nghiên cứu liên quan.
	\item[-] Chương 2: Trình bày về tiêu chí vết - tiếu chí lựa chọn đặc trưng. Sau đó chứng minh các kết quả và cuối cùng trình bày Thuật toán PSFT - thuật toán lựa chọn đặc trưng song song dựa trên tiêu chí vết.
	\item[-] Chương 3: Trình bày về các tập dữ liệu được sử dụng khi làm thực nghiệm. Thông tin về phần cứng được sử dụng và thảo luận kết quả thực nghiệm.
 	\item[-] Cuối cùng: Phần kết luận và danh mục các tài liệu tham khảo.
	
\end{itemize}

\noindent {\bf 2. NHỮNG KẾT QUẢ MỚI CỦA LUẬN VĂN:} 

Xây dựng một thuật toán lựa chọn đặc trưng song song dựa trên tiêu chí vết (thuật toán PFST) trên các tập dữ liệu lớn. Thuật toán sử dụng tiêu chí vết để đánh giá các đặc trưng tốt. Thực nghiệm cho thấy thuật toán cho kết quả tốt không chỉ về mặt hiệu suất của mô hình LDA cho bài toán phân loại mà còn tiết kiệm chi phí tính toán, khi thời gian chạy và đạt kết quả nhanh chóng.

\noindent {\bf 3. CÁC ỨNG DỤNG/ KHẢ NĂNG ỨNG DỤNG TRONG THỰC TIỄN HAY NHỮNG VẤN ĐỀ CÒN BỎ NGỎ CẦN TIẾP TỤC NGHIÊN CỨU:}
Kết quả luận văn có thể được dùng để lựa chọn đặc trưng trên các tập dữ liệu thực tế. Tuy nhiên, một trong những nhược điểm của PFST là tiêu chí vết chỉ có thể sử dụng cho các đặc trưng liên tục. Do đó, trong tương lại, chúng tôi vẫn sẽ nỗ lực làm việc, và hi vọng sẽ khám phá ra cách mở rộng cho các trường hợp còn lại trong bài toán phân loại.
\newpage

\begin{tabular}{ccccc}
 {\bf CÁN BỘ HƯỚNG DẪN}	&&&& {\bf HỌC VIÊN CAO HỌC} \\
 (Ký tên, họ tên) &&&& (Ký tên, họ tên) \\
\end{tabular}

\vspace{3cm}

\begin{center}
{\bf XÁC NHẬN CỦA CƠ SỞ ĐÀO TẠO}

{\bf HIỆU TRƯỞNG}
\end{center}
\thispagestyle{empty}