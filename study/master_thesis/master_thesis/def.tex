\def\Z{\mathbb{Z}} 
\def\Q{\mathbb{Q}}
\def\R{\mathbb{R}}
\def\C{\mathbb{C}}
%%----------Định nghĩa lại cho dấu tương đương và dấu suy ra.
\def\iffs{\Leftrightarrow} \def\suy{\Rightarrow} 
%---------------------------------------------------------
\newcommand{\id}[1]{{\em #1}\index{#1}} %Tạo từ chỉ mục và in nghiêng từ cần làm chỉ mục
%%---------------------------------------------------
%%------------Định nghĩa lại môi trường Định lý, Định nghĩa, Mệnh đề, ...
\newtheorem{theorem}{Định lý}[section]
\newtheorem{proposition} [theorem]{Mệnh đề}
\newtheorem{corollary} [theorem]{Hệ quả}
\newtheorem{lemma} [theorem]{Bổ đề}
%\newtheorem{gt} [theorem]{Giả thiết}
\theoremstyle{definition}\newtheorem{definition}[theorem]{Định nghĩa}
\theoremstyle{definition}\newtheorem{note}[theorem]{Chú ý}
\theoremstyle{definition}\newtheorem{remark}[theorem]{Nhận xét}
\newtheorem{example}[theorem]{Ví dụ}
%%----- Tạo môi trường có tên Ví dụ và đánh số cho ví dụ (cùng cấp với chapter)----- 
\newcounter{danhsovidu}[chapter] %tao o dem moi de danh so cho vi du
\newenvironment{vidu}{\stepcounter{danhsovidu}\par\noindent{\bfseries Ví dụ \thesection.\thedanhsovidu.}}{\par}

\newcommand{\eqand}[1]{\left\{\begin{array}{l}#1\end{array}\right.}
\newcommand{\eqor}[1]{\left[\begin{array}{l}#1\end{array}\right.}
\newcommand{\vt}[1]{\overrightarrow{#1}}
%%---------------------------------
%%
%%----------Tự tạo header và footer riêng-----------------------%%
\makeatletter
\newcommand{\ps@myplain}{%khai báo kiểu định dạng mới myplain
\renewcommand{\@oddhead}{\textit{Luận văn thạc sĩ toán học - Cơ sở Toán cho Chuyên ngành Khoa học dữ liệu}\dotfill } 
%tạo header trang lẻ
\renewcommand{\@evenhead}{ \dotfill \textit{Parallel feature selection based on the trace ratio criterion}} 
%tạo header trang chẵn
 \renewcommand{\@oddfoot}{\mbox{\emph{Phan Thành Nhân - Chuyên ngành Khoa học dữ liệu K30}}\dotfill Trang \thepage} 
% tạo footer trang lẻ
\renewcommand{\@evenfoot}{\@oddfoot}} 
% tạo footer trang chẵn giống footer trang lẻ
\makeatother
%%----------------End of file def.tex
