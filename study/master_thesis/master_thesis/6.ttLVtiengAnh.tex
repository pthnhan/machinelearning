\thispagestyle{empty}
\begin{center}
	\LARGE{\textbf{THESIS INFORMATION}}
	\addcontentsline{toc}{chapter}{Trang thông tin luận văn tiếng Anh}
\end{center}

\noindent Thesis title: Parallel feature selection based on the trace ratio criterion
\\
Speciality: Data science
\\
Code:  8480109
\\
Name of Master Student: Phan Thành Nhân
\\
Academic year: 2020 - 2022
\\
Supervisor:  Dr. Nguyen Thi Thu
\\
At: VNUHCM - University of Science\\
\\
{\bf 1. SUMMARY:}
\begin{itemize}
	\item[-] Chapter 1: An overview of feature selection and related works.
	\item[-] Chapter 2: Present trace criterion as a feature selection criterion. Prove the results and present parallel Feature Selection  using  Trace  criterion (PFST Algorithm).
	\item[-] Chapter 3: Presentation of the experiment's data sets. Discussion of the experimental findings and information on the hardware utilized.
        \item[-] Finally: Conclusion and list of references.
\end{itemize}

\noindent {\bf 2. NOVELTY OF THESIS:}  The main results in this thesis are as follows. 

Presents a novel parallel feature selection approach for classification, called PFST, for feature selection on large-scale datasets. The method uses the trace criterion. The experiments show that our PFST method achieves good accuracy using the LDA model and less running time to select relevant features.

\noindent {\bf 3. APPLICATIONS/ APPLICABILITY/ PERSPECTIVE:}

The results of this thesis can be used to select the features of the real data sets. However, one of the drawbacks of the algorithm is that the trace criterion can only be used for continuous features. Therefore, in the future, it would be desirable to explore how to extend this work to the case where there are categorical features in the model as well.
\newpage

\begin{center}
\begin{tabular}{ccccccccccccc}
	{\bf SUPERVISOR}	&&&&&&&&&&&& {\bf Master STUDENT} \\
\end{tabular}
\end{center}

\vspace{3cm}

\begin{center}
	{\bf CERTIFICATION}\\ 
	{\bf UNIVERSITY OF SCIENCE}\\
	{\bf PRESIDENT}
\end{center}
\thispagestyle{empty}