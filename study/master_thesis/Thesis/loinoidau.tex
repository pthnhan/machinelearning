\chapter*{Lời nói đầu}
\addcontentsline{toc}{chapter}{Lời nói đầu}

Trong kỷ nguyên của dữ liệu lớn, sự phát triển của dữ liệu đặt ra những thách thức đối với việc quản lý và sử dụng dữ liệu hiệu quả. Ví dụ: một tập dữ liệu gen có thể chứa hàng trăm nghìn đặc điểm \cite{guyon2007competitive}. Do đó, việc xử lý trực tiếp các tập dữ liệu như vậy có thể gặp sự khó khăn về kích thước. Hơn nữa, các đặc trưng dư thừa có thể làm giảm hiệu suất học tập của các thuật toán phân loại. Để giải quyết vấn đề này, nhiều kỹ thuật giảm chiều dữ liệu đã được phát triển \cite{melab2006grid,de2006parallelizing,garcia2006parallel,guillen2009efficient,lopez2006solving} và chúng được phân loại thành các phương pháp trích xuất đặc trưng hoặc lựa chọn đặc trưng \cite{liu2012feature, kumar2014feature}. Kỹ thuật trích xuất đặc trưng (ví dụ: Phân tích thành phần chính (Principal Component Analysis - PCA) \cite{johnson2002applied}, Phân tích phân biệt tuyến tính (Linear Discriminant Analysis - LDA) \cite{johnson2002applied}) liên quan đến việc chiếu dữ liệu vào một không gian đối tượng mới với số chiều nhỏ hơn thông qua một vài bước biến đổi tuyến tính hoặc phi tuyến từ các đặc trưng gốc. Tuy nhiên, điều này tạo ra một loạt các đặc trưng mới mà không thể diễn giải trực tiếp. Hơn nữa, vì những cách tiếp cận đó sử dụng tất cả các đặc trưng có sẵn trong quá trình trích xuất đặc trưng nên nó không giúp giảm chi phí lưu trữ dữ liệu và chi phí thu thập dữ liệu trong tương lai. Mặt khác, các phương pháp lựa chọn đặc trưng (\cite{sinaga2021entropy, james2013introduction},\ldots) chỉ chọn một tập hợp con các đặc trưng hữu ích để xây dựng mô hình. Do đó, điều này giúp giữ các tính chất của các đặc trưng ban đầu trong khi giảm chi phí lưu trữ và thu thập dữ liệu trong tương lai bằng cách loại bỏ các đặc trưng không liên quan. Tuy nhiên, các dữ liệu từ các lĩnh vực khác nhau như khai thác văn bản, phân tích kinh doanh và sinh học, thường được đo bằng gigabyte hoặc terabyte với hàng triệu đặc trưng \cite{bolon2015feature, li2017feature}. Ví dụ: tập dữ liệu Amazon Review \cite{ni2019justifying} là tập dữ liệu 34 gigabyte. Trong những trường hợp như vậy, hiệu suất của các kỹ thuật lựa chọn đặc trưng mới nhất có thể bị ảnh hưởng  \cite{li2017feature}. Điều này là do không gian tìm kiếm cho một tập hợp con các đặc trưng hữu ích bị tăng lên đáng kể. Một cách để giải quyết vấn đề này là sử dụng tính toán song song, cho phép sử dụng tốt hơn tài nguyên tính toán của máy tính bằng cách phân vùng dữ liệu và chạy các lựa chọn đặc trưng trên nhiều lõi cùng một lúc.

Trong đề tài luận văn cao học này, chúng tôi phát triển một phương pháp \lq\lq Sử dụng tính toán song song để lựa chọn đặc trưng dựa trên tiêu chí tỉ lệ vết (Parallel feature selection based on the trace ratio criterion - PFST)\rq\rq cho bài toán phân loại. Tiêu chí được sử dụng là một thước đo về khả năng tách lớp được sử dụng trong LDA, để đánh giá tính hữu dụng của đặc trưng. Dựa trên tiêu chí này, PFST sẽ nhanh chóng tìm thấy các đặc trưng quan trọng từ một tập hợp các đặc trưng gốc của tập dữ liệu lớn bằng cách sử dụng sức mạnh của tính toán song song để thực hiện lựa chọn đặc trưng và loại bỏ các đặc trưng dư thừa. Sau khi các đặc trưng quan trọng nhất được đưa vào mô hình, chúng tôi đánh giá phương pháp thông qua các thử nghiệm khác nhau bằng cách sử dụng LDA làm mô hình phân loại. Thực nghiệm cho thấy rằng phương pháp của chúng tôi có thể chọn ra một tập hợp con nhỏ các đặc trưng trong thời gian ngắn, tiết kiệm thời gian hơn so với một số phương pháp khác. Ngoài ra, độ chính xác khi phân loại dựa trên các đặc trưng được lựa chọn bằng PFST cũng đạt độ chính xác cao và tốt hơn các phương pháp khác, và tốt hơn việc phân loại dựa trên tập các đặc trưng gốc ban đầu.

Nội dung khóa luận này bao gồm 3 chương. Trong đó,
\begin{description}
	\item[Chương 1:] Chương này giới thiệu tổng quan về lựa chọn đặc trưng và các nghiên cứu liên quan.
	\item[Chương 2:] Chương này trình bày về tiêu chí vết - tiếu chí lựa chọn đặc trưng. Thuật toán PSFT - thuật toán lựa chọn đặc trưng song song dựa trên tiêu chí vết.
	\item[Chương 3:] Chương này trình bày về các tập dữ liệu được sử dụng khi làm thực nghiệm. Thông tin về phần cứng được sử dụng và thảo luận kết quả
	\item[Cuối cùng:] Phần kết luận và danh mục các tài liệu tham khảo.
\end{description}
