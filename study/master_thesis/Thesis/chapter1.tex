\chapter{Tổng quan}\label{chapter:tq}

\section{Lý do chọn đề tài}
Trong kỷ nguyên của dữ liệu lớn, sự phát triển của dữ liệu đặt ra những thách thức đối với việc quản lý và sử dụng dữ liệu hiệu quả. Ví dụ: một tập dữ liệu gen có thể chứa hàng trăm nghìn đặc điểm \cite{guyon2007competitive}. Do đó, việc xử lý trực tiếp các tập dữ liệu như vậy có thể gặp sự khó khăn về kích thước. Hơn nữa, các đặc trưng dư thừa có thể làm giảm hiệu suất học tập của các thuật toán phân loại. Để giải quyết vấn đề này, nhiều kỹ thuật giảm chiều dữ liệu đã được phát triển \cite{melab2006grid,de2006parallelizing,garcia2006parallel,guillen2009efficient,lopez2006solving} và chúng được phân loại thành các phương pháp trích xuất đặc trưng hoặc lựa chọn đặc trưng \cite{liu2012feature, kumar2014feature}. Kỹ thuật trích xuất đặc trưng (ví dụ: Phân tích thành phần chính (Principal Component Analysis - PCA) \cite{johnson2002applied}, Phân tích phân biệt tuyến tính (Linear Discriminant Analysis - LDA) \cite{johnson2002applied}) liên quan đến việc chiếu dữ liệu vào một không gian đối tượng mới với số chiều nhỏ hơn thông qua một vài bước biến đổi tuyến tính hoặc phi tuyến từ các đặc trưng gốc. Tuy nhiên, điều này tạo ra một loạt các đặc trưng mới mà không thể diễn giải trực tiếp. Hơn nữa, vì những cách tiếp cận đó sử dụng tất cả các đặc trưng có sẵn trong quá trình trích xuất đặc trưng nên nó không giúp giảm chi phí lưu trữ dữ liệu và chi phí thu thập dữ liệu trong tương lai. Mặt khác, các phương pháp lựa chọn đặc trưng (\cite{sinaga2021entropy, james2013introduction},\ldots) chỉ chọn một tập hợp con các đặc trưng hữu ích để xây dựng mô hình. Do đó, điều này giúp giữ các tính chất của các đặc trưng ban đầu trong khi giảm chi phí lưu trữ và thu thập dữ liệu trong tương lai bằng cách loại bỏ các đặc trưng không liên quan. Tuy nhiên, các dữ liệu từ các lĩnh vực khác nhau như khai thác văn bản, phân tích kinh doanh và sinh học, thường được đo bằng gigabyte hoặc terabyte với hàng triệu đặc trưng \cite{bolon2015feature, li2017feature}. Ví dụ: tập dữ liệu Amazon Review \cite{ni2019justifying} là tập dữ liệu 34 gigabyte. Trong những trường hợp như vậy, hiệu suất của các kỹ thuật lựa chọn đặc trưng mới nhất có thể bị ảnh hưởng  \cite{li2017feature}. Điều này là do không gian tìm kiếm cho một tập hợp con các đặc trưng hữu ích bị tăng lên đáng kể. Một cách để giải quyết vấn đề này là sử dụng tính toán song song, cho phép sử dụng tốt hơn tài nguyên tính toán của máy tính bằng cách phân vùng dữ liệu và chạy các lựa chọn đặc trưng trên nhiều lõi cùng một lúc.

Từ những lý do trên, trong đề tài luận văn cao học này, chúng tôi nghiên cứu và trình bày một phương pháp \lq\lq Sử dụng tính toán song song để lựa chọn đặc trưng dựa trên tiêu chí tỉ lệ vết (Parallel feature selection based on the trace ratio criterion - PFST)\rq\rq cho bài toán phân loại. Tiêu chí đánh giá mức độ hữu dụng của đặc trưng được sử dụng là một thước đo về khả năng tách lớp được sử dụng trong bài toán Linear discriminant analysis (LDA - chúng tôi sẽ trình bày bài toán này bên dưới). Dựa trên tiêu chí này, PFST sẽ nhanh chóng tìm thấy các đặc trưng quan trọng từ một tập hợp các đặc trưng gốc của tập dữ liệu lớn và loại bỏ các đặc trưng dư thừa. Bên cạnh đó, chúng tôi thiết lập và sử dụng tính toán song song để tối ưu nguồn tài nguyên của máy tính nhằm tăng hiệu suất tính toán và thời gian chạy. Cuối cùng, chúng tôi sẽ sử dụng LDA làm mô hình phân loại để đánh giá độ hiệu quả của thuật toán PFST. Bên cạnh đó, chúng tôi cũng so sánh với một số thuật toán khác. Kết quả thu được rất tốt, PFST có thời gian chạy và có độ chính xác cao hơn các phương pháp khác. Bên cạnh đó, tập đặc trưng thu được từ PFST cũng cho ra kết quả tốt hơn nếu sử dụng toàn bộ đặc trưng ban đầu.

\section{Các công trình nghiên cứu khoa học có liên quan}
Đối mặt với các thách thức về kích thước của dữ liệu ngày càng tăng, đã có nhiều nỗ lực trong hướng nghiên cứu về lựa chọn đặc trưng để phát triển các kỹ thuật mới. Bên cạnh các phương pháp lai để kết hợp các chiến lược lựa chọn đặc trưng khác nhau \cite{saeys2007review,ang2015supervised}, hầu hết các phương pháp lựa chọn đặc trưng có thể được chia thành ba loại.

Đầu tiên, cách tiếp cận \lq\lq wrapper\rq\rq~ dựa trên hiệu suât của một thuật toán học máy cụ thể để đánh giá tầm quan trọng của các đặc trưng được chọn. Một phương pháp \lq\lq wrapper\rq\rq~ điển hình sẽ tìm kiếm một tập con các đặc trưng dựa trên một thuật toán học máy trước, sau đó sẽ đánh giá chúng. Các bước này được lặp lại cho đến khi thỏa mãn một số tiêu chí dừng. Các phương pháp trong loại này thường rất tốn kém về chi phí tính toán vì việc đánh giá tập con các đặc trưng yêu cầu nhiều lần lặp lại. Mặc dùng nhiều các tiếp cận tìm kiếm được đề xuất chẳng hạn như thuật toán tìm kiếm best-first \cite{arai2016unsupervised} và thuật toán di chuyền (genetic) \cite{goldberg1988genetic}. Tuy nhiên, việc sử dụng các thuật toán này cho dữ liệu nhiều chiều vẫn không cho thấy sử cải thiện về chi phí tính toán.

Thứ hai, cách tiếp cận \lq\lq filter\rq\rq~ bao gồm các kỹ thuật đánh giá các tập hợp con đặc bằng việc xếp hạng với một số tiêu chí như tiêu chí thông tin \cite{nguyen2014efficiency, shishkin2016efficiency}, khả năng tái tạo \cite{farahat2011efficiency, masaeli2010convex}. Các phương pháp này chọn các đặc trưng độc lập với thuật toán học máy và thường hiệu quả hơn về chi phí tính toán so với các phương pháp \lq\lq wrapper\rq\rq~\cite{li2017feature}. Tuy nhiên, vì không được tối ưu hóa cho bất kỳ thuật toán học máy mục tiêu nào, nên chúng có thể không tối ưu cho một tuật toán học máy cụ thể.

Thứ ba, các phương pháp \lq\lq embedded\rq\rq~ sử dụng các tiêu chí độc lập để tìm ra tập con tối ưu cho một tập hợp nhất định. Sau đó, một thuật toán học máy được sử dụng để lựa chọn tập con tối ưu cuối cùng trong số các tập con tối ưu trên các tập hợp khác nhau. Vì thế, chúng hiệu quả hơn về chi phí tính toán so với các phương pháp \lq\lq wrapper\rq\rq~ vì chúng không đánh giá đặc trưng dựa trên việc lặp lại các tập con đặc trưng. Ngoài ra, chúng cũng được huấn luyện từ các thuật toán học máy. Vì thế, chúng có thể được coi như sự đánh đổi giữa phương pháp \lq\lq filter\rq\rq~ và phương pháp \lq\lq wrapper\rq\rq \cite{li2017feature}.

Mặc dù, cho đến này, các nhà khoa học đã nỗ lực rất nhiều trong hướng nghiên cứu lựa chọn đặc trưng, nhưng dữ liệu từ các trường, các ngành khác nhau có thể quá phong phú ngay cả đối với các phương pháp \lq\lq filter\rq\rq hiệu quả về chi phí tính toán. Điều này đã thúc đẩy nhiều nghiên cứu khác trong việc lựa chọn đặc trưng song song. Một số phương pháp đã được đề xuất trong \cite{melab2006grid,de2006parallelizing,garcia2006parallel,guillen2009efficiency,lopez2006solve} sử dụng quy trình xử lý song song để đánh giá đồng thời nhiều đặc trưng. Tuy nhiên, các thuật toán này yêu cầu quyên truy cập vào toàn bộ dữ liệu. Mặc khác, trong trong \cite{singh2009parallel}, các tác giả đã đề xuất một thuật toán lựa chọn đặc trưng song song cho hồi quy logistic dựa trên framework MapReduce và các đặc trưng được đánh giá thông qua hàm mục tiêu của mô hình hồi quy logistic. Trong khi đó, các tác giả của bài báo \cite{tsamardinos2019greedy} đã đề xuất \textit{Song song, Tiến–Lùi với thuật toán Tỉa (Parallel, Forward–Backward with Pruning algorithm)} (PFBP) để lựa chọn đặc trưng bằng cách bỏ sớm một số đặc trưng trong các lần lặp lại tiếp theo và sớm trả ra kết quả đặc trưng tốt nhất trong mỗi lần lặp. Tuy nhiên, các tiếp cận này yêu cầu tính toán bootstrap của p-giá trị, rất tốn kém chi phí tính toán. Trong bài báo \cite{zhao2013massively}, Zhao và các cộng sự đã giới thiệu một thuật toán lựa chọn đặc trưng song song để chọn các đặc trưng dựa trên khả năng giải thích phương sai của dữ liệu. Tuy nhiên, theo các tiếp cận của họ, việc xác định số lượng các đặc trưng trong mô hình dựa trên việc chuyển đổi các nhãn phân loại thành các giá trị số và sử dụng tổng bình phương sai số. Việc tính tổng bình phương sai số đòi hỏi phải điều chỉnh mô hình và do đó thuật toán vẫn còn tốn kém rất nhiều chi phí tính toán.