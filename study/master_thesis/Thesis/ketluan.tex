\chapter*{Kết luận}
\addcontentsline{toc}{chapter}{Kết luận}
Qua khóa luận này, chúng tôi đã trình bày phương pháp PFST, một cách tiếp cận mới về việc lựa chọn song song các đặc trưng từ các tập dữ liệu lớn trong bài toán phân loại. Phương pháp của chúng tôi đã sử dụng tiêu chí vết, là tiêu chí có các thuộc tính để đánh giá các đặc trưng tốt. Chúng tôi ước lượng các tiếp cận này thông qua nhiều thực nghiệm và các tập dữ liệu khác nhau. Chúng tôi sử dụng LDA làm mô hình phân loại trên các đặc trưng được chọn. Thực nghiệm cho thấy phương pháp của chúng tôi có thể chọn ra các đặc trưng tốt với thời gian ngắn hơn khi so với các phương pháp tiếp cận khác. Hơn nữa, phân loại dựa trên các đặc trưng chọn được bằng PFST cũng thu được độ chính xác tốt hơn các phương pháp khác (tốt hơn bốn trong năm tập dữ liệu) và tốt hơn khi chạy mô hình phân loại với toàn bộ đặc trưng. Tuy nhiên, một trong những nhược điểm của PFST là tiêu chí vết chỉ có thể sử dụng cho các đặc trưng liên tục. Do đó, trong tương lại, chúng tôi vẫn sẽ nỗ lực làm việc, và hi vọng sẽ khám phá ra cách mở rộng cho các trường hợp còn lại trong bài toán phân loại.

Ngoài ra, chúng tôi cũng đã xây dựng thuật toán như một thư viện trong python. Người dùng có thể tìm thấy và cài đặt trong github sau: \url{https://github.com/pthnhan/PFST}