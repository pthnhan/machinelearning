\documentclass[a4paper,openright]{book}
\usepackage{amsmath,amssymb,tasks}
\usepackage{fancyhdr}
\usepackage{enumerate}
\usepackage{tkz-euclide}
\usepackage{amssymb,amsthm,makeidx,enumerate,amsmath}
\usepackage{mathrsfs} 
\usepackage{pgf,tikz,pstricks-add}
\usetikzlibrary{arrows}
\usepackage[utf8]{inputenc}
\usepackage[utf8]{vietnam}
\newtheorem{theorem}{Định lí}
\usepackage[top=2.5cm, bottom=2.5cm, left=2.5cm, right=2.5cm] {geometry}
\setlength{\parindent}{0.6cm}
\setlength{\parskip}{0.2cm}
\makeindex
\usepackage{mathpazo}
\usepackage{subfig}
\usepackage{xhfill}
\usepackage{cases}
\usepackage{commath}
\usepackage{xcolor}
\usepackage{hyperref}
\hypersetup{
	colorlinks=true,
	linkcolor=blue,
	filecolor=magenta,      
	urlcolor=cyan,
}
\usepackage{arydshln}
\usepackage{tikz,tkz-euclide}
%\usetkzobj{all}
\usepackage{indentfirst}
\providecommand{\abs}[1]{\lvert#1\rvert}
\providecommand{\norm}[1]{\lVert#1\rVert}
\providecommand{\vt}[1]{\left\langle#1\right\rangle}
\newcommand{\function}[3]{#1 \hspace{0.4cm} : \hspace{0.4cm} \mathbb{#2} \rightarrow \mathbb{#3}}
\newcommand{\hoac}[1]{\left[\begin{aligned}#1\end{aligned}\right.}
\newcommand{\heva}[1]{\left\{\begin{aligned}#1\end{aligned}\right.}
\newcommand{\ve}{vectơ }
\newcommand{\C}{\mathbb{C}}
\newcommand{\R}{\mathbb{R}}
\newcommand{\Z}{\mathbb{Z}}
\newcommand{\N}{\mathbb{N}}
\newcommand{\I}{\mathbb{I}}
\newcommand{\Q}{\mathbb{Q}}
\newcommand{\E}{\mathbb{E}}
\newtheorem{DL}{Định lý}[section]
\newtheorem{BD}[DL]{Bổ Đề}
\newtheorem{HQ}[DL]{Hệ Quả}
\newtheorem{TC}[DL]{Tính chất}
\newtheorem{VD}[DL]{Ví dụ}
\newtheorem{NX}[DL]{Nhận xét}
\theoremstyle{definition}
\newtheorem{DN}[DL]{Định nghĩa}
%\newtheorem{NX}[DL]{Nhận xét}
\theoremstyle{remark}
\newcommand{\enter}{\\[6pt]}
\newcommand{\XongCM}{\\[-30pt]
	\begin{flushright}
		$\blacksquare$
\end{flushright}}
\setlength{\parskip}{0.8em}
\renewcommand{\baselinestretch}{1.2}
\usepackage{mathpazo}
\usepackage{scrextend}
\changefontsizes{12pt}
 \allowdisplaybreaks
\usepackage{multicol}
\newcommand{\lietke}[2]{\begin{listEX}[#1]#2\end{listEX}}
\numberwithin{equation}{section}
\usepackage{afterpage}

\newcommand\blankpage{
	\null
	\thispagestyle{empty}
%	\addtocounter{page}{-1}
	\newpage
}

%\usepackage[style=ieee]{biblatex}
%\addbibresource{bbli.bib}

\usepackage{cite}
\usepackage{amsmath,amssymb,amsfonts}
\usepackage{algorithm}
\makeatletter
\renewcommand{\ALG@name}{Thuật toán}
\makeatother
\usepackage{algpseudocode}
\usepackage{graphicx}
\usepackage{textcomp}
\usepackage{algpseudocode}
% \newtheorem{proof}{Proof}[section]
%\newtheorem{alg}{Thuật toán}[chapter]
\usepackage{multirow}
\usepackage{caption}
\def\BibTeX{{\rm B\kern-.05em{\sc i\kern-.025em b}\kern-.08em
		T\kern-.1667em\lower.7ex\hbox{E}\kern-.125emX}}

\captionsetup[algorithm]{
	labelfont = bf,
	labelsep = period
}

\makeatletter
\newenvironment{breakablealgorithm}
{% \begin{breakablealgorithm}
%		\begin{center}
			\refstepcounter{algorithm}% New algorithm
			~\par
			\hrule height.8pt depth0pt \kern2pt
			\hrule height.8pt depth0pt \kern2pt% \@fs@pre for \@fs@ruled
			\renewcommand{\caption}[2][\relax]{% Make a new \caption
				{\raggedright\textbf{\fname@algorithm~\thealgorithm.} ##2\par}%
				\ifx\relax##1\relax % #1 is \relax
				\addcontentsline{loa}{algorithm}{\protect\numberline{\thealgorithm}##2}%
				\else % #1 is not \relax
				\addcontentsline{loa}{algorithm}{\protect\numberline{\thealgorithm}##1}%
				\fi
%				\kern2pt\hrule\kern1pt
				\vspace{1pt}
				\hrule height.8pt depth0pt \kern2pt
				\hrule height.8pt depth0pt \kern2pt
			}
		}{% \end{breakablealgorithm}
		\kern2pt\hrule\relax% \@fs@post for \@fs@ruled
%	\end{center}
}
\makeatother

\DeclareCaptionFormat{algor}{%
	\hrulefill\par\offinterlineskip\vskip1pt%
	\textbf{#1#2}#3\offinterlineskip\hrulefill}
\DeclareCaptionStyle{algori}{singlelinecheck=off,format=algor,labelsep=space}
\captionsetup[algorithm]{style=algori}

\usepackage{listings}
\begin{document}


\makeatletter
%\setbeamertemplate{headline}{}
\setbeamertemplate{footline}
{
	\leavevmode%
	\hbox{%
		\begin{beamercolorbox}[wd=.25\paperwidth,ht=2.25ex,dp=1ex,center]{author in head/foot}%
			\usebeamerfont{author in head/foot} \insertshortauthor
		\end{beamercolorbox}%  -uytrfder      jo  
		\begin{beamercolorbox}[wd=.5\paperwidth,ht=2.25ex,dp=1ex,center]{title in head/foot}%
			\usebeamerfont{title in head/foot}\insertshorttitle
		\end{beamercolorbox}%
		\begin{beamercolorbox}[wd=.25\paperwidth,ht=2.25ex,dp=1ex,right]{date in head/foot}%
			\usebeamerfont{date in head/foot}\insertshortdate \hspace*{1.5em}
			%	Trang \insertframenumber\hspace*{2ex} 
			%	\insertframenumber/39 \hspace*{2ex}
			\insertframenumber/\inserttotalframenumber \hspace*{2ex}
		\end{beamercolorbox}}%
		\vskip0pt%
	}
	\makeatother

	





\newcommand{\doo}[1]{\textcolor{red}{#1}}
\newcommand{\mdo}[1]{\textcolor{red}{\bm{#1}}}
\newcommand{\tdo}[1]{\textcolor{red}{\textbf{#1}}}
\newcommand{\nau}[1]{\textcolor{brown}{#1}}
\newcommand{\mnau}[1]{\textcolor{brown}{\bm{#1}}}
\newcommand{\xanh}[1]{\textcolor{green}{#1}}
\newcommand{\duong}[1]{\textcolor{blue}{#1}}
\newcommand{\mduong}[1]{\textcolor{blue}{\bm{#1}}}
\newcommand{\tduong}[1]{\textcolor{blue}{\textbf{#1}}}
\newcommand{\bich}[1]{\textcolor{cyan}{#1}}
\newcommand{\mbich}[1]{\textcolor{cyan}{\bm{#1}}}
\newcommand{\tbich}[1]{\textcolor{cyan}{\textbf{#1}}}
\newcommand{\hong}[1]{\textcolor{magenta}{#1}}
\newcommand{\vang}[1]{\textcolor{yellow}{#1}}
\newcommand{\pt}[1]{\textcolor{green}{\mathbf{#1}}}

%\newcommand{\dn}[1]{\textcolor{red}{\textbf {\textit  {#1}}}}

\newcommand{\dn}[1]{\textcolor{ForestGreen}{\textbf {\textit  {#1}}}}  %Màu dùng có các định nghĩa

%\newcommand{\dnt}[1]{\textcolor{blue}{\bm{#1}}} 
\newcommand{\dnt}[1]{\textcolor{RoyalBlue}{\bm{#1}}}   %Màu dùng cho công thức toán đặc biệt 


\newenvironment{test}{\textbf{Thử nghiệm}}{}
\def\K{\mathbb{R}}
\def\R{\mathbb{R}}
\def\N{\mathbb{N}}
\def\Q{\mathbb{Q}}
\def\C{\mathbb{C}}
\def\Z{\mathbb{Z}} 
\def\F{\mathbb{F}}
\def\rank{\mathrm{rank}}
\def\Id{\mathrm{Id}}






\newcommand{\cm}{\notab{\mau{Chứng minh.}} }
\newcommand{\gt}{\notab{\mau{Giải thích.}} }
\newcommand{\giai}{\notab{\mau{Giải.}} }
\newcommand{\gthich}{\notab{\mau{Giải thích.}} }
%\newcommand{\cm}{\notab{\textcolor{brown}{\bf Chứng minh.}} } 
%\newcommand{\giai}{\notab{\textcolor{brown}{\bf Giải.}} }


 
\newcommand{\dpcm}{ \hfill \nau{\rule{2.5mm}{2.5mm}}}
\newcommand{\giua}[1]{\begin{center}#1\end{center}}






\newenvironment{DL}{ \begin{block}{} \textbf{\duong{Định lý.}} \begin{itshape}} {\end{itshape}\end{block}}
\newenvironment{MD}{ \begin{block}{} \textbf{\duong{Mệnh đề.}} \begin{itshape}} {\end{itshape}\end{block}}
\newenvironment{HQ}{ \begin{block}{} \textbf{\duong{Hệ quả.}} \begin{itshape}} {\end{itshape}\end{block}}
\newenvironment{BD}{ \begin{block}{} \textbf{\duong{Bổ đề.}} \begin{itshape}} {\end{itshape}\end{block}}
\newenvironment{TC}{ \begin{block}{} \textbf{\duong{Tính chất.}} \begin{itshape}} {\end{itshape}\end{block}}
\newenvironment{GT}{ \begin{block}{} \textbf{\duong{Giả thiết.}} \begin{itshape}} {\end{itshape}\end{block}}
\newenvironment{NX}{ \begin{block}{} \textbf{\duong{Nhận xét.}} \begin{itshape}} {\end{itshape}\end{block}}



\newenvironment{prob}{ \begin{block}{\textbf{Problem}}} {\end{block}}

\newenvironment{moti}{ \begin{block}{\textbf{Motivation}}} {\end{block}}

\newenvironment{exam}{ \begin{block}{\textbf{Example}}}{\end{block}}

\newenvironment{algo}{ \begin{block}{\textbf{Algorithm}}} {\end{block}}

\newenvironment{theo}{ \begin{block}{\textbf{Theorem}} \begin{itshape}} {\end{itshape}\end{block}}

\def\question {\textbf{\duong{Question. }}}















\begin{titlepage}
	\begin{center}
		{\bf  ĐẠI HỌC QUỐC GIA TP. HỒ CHÍ MINH}\\
		{\bf TRƯỜNG ĐẠI HỌC KHOA HỌC TỰ NHIÊN}\\
		\hfill
		
		\vspace*{2cm}
		
		{\large\bf  PHAN THÀNH NHÂN - 20C29012}
		
		\vspace*{3cm}
		
		{\huge\bf SỬ DỤNG TÍNH TOÁN SONG SONG\\
			ĐỂ LỰA CHỌN ĐẶC TRƯNG\\
			DỰA TRÊN TIÊU CHÍ TỈ LỆ VẾT	
	}
		
		\vspace*{3cm}
		
		{\large\bf LUẬN VĂN THẠC SĨ}
		
		{\bf Tp. Hồ Chí Minh - 2022}
	\end{center}
	
\end{titlepage}


\afterpage{\blankpage}
\newpage
\thispagestyle{empty}
\begin{center}
	{\bf  ĐẠI HỌC QUỐC GIA TP. HỒ CHÍ MINH}\\
	{\bf TRƯỜNG ĐẠI HỌC KHOA HỌC TỰ NHIÊN}
	
	\vspace*{1.5cm}
	
	{\bf  PHAN THÀNH NHÂN - 20C29012}
	
	\vspace*{1.5cm}
	
	{\huge\bf SỬ DỤNG TÍNH TOÁN SONG SONG\\
	ĐỂ LỰA CHỌN ĐẶC TRƯNG\\
	DỰA TRÊN TIÊU CHÍ TỈ LỆ VẾT	
}
	
	\vspace*{1.5cm}
	
	{\large\bf LUẬN VĂN THẠC SĨ}\\[20pt]
	CHUYÊN NGÀNH CƠ SỞ TOÁN CHO KHOA HỌC DỮ LIỆU
	
	\vspace*{1.5cm}
	
	GIẢNG VIÊN HƯỚNG DẪN KHOA HỌC
	
	{\bf TS. Nguyễn Thị Thu}
	
	\vfill
	{\bf Tp. Hồ Chí Minh - 2022}
\end{center}


\afterpage{\blankpage}
\vspace*{1cm}
{\Large\textbf{Lời cảm ơn}}

%Lời đầu tiên, tôi xin chân thành gửi lòng biết ơn sâu sắc nhất đến Thầy Lê Văn Luyện, người đã hướng dẫn cho tôi trong suốt quá trình làm khóa luận, cũng như đã tin tưởng đặt cho tôi những viên gạch đầu tiên để tôi có thể hình dung được cách thức nghiên cứu Toán học, đặc biệt là tiếp tục phát triển những ý tưởng của khóa luận này trong tương lai.
%
%Qua đây, tôi cũng xin gửi lời cảm ơn đến các thành viên nhóm seminar mật mã do thầy Lê Văn Luyện và Dương Hoàng Dũng tổ chức, đã giúp tôi từng bước một để tìm hiểu về Toán - Mã, cho tôi những cái nhìn tổng quan, giúp tôi chọn được đề tài thích hợp và giải thích chia sẻ tài liệu để tôi có thể hoàn thành được khóa luận này.
%
%Ngoài ra, tôi cũng xin bày tỏ lòng tri ân đến các giảng viên của khoa Toán - Tin học, trường Đại học Khoa học Tự nhiên, Tp. Hồ Chí Minh, đặc biệt là các thầy trong bộ môn Đại số, đã dạy dỗ và giúp đỡ tôi trong thời gian qua, giúp tôi có cái nhìn toàn cảnh về các kiến thức toán học. Hơn nữa, các thầy đã giúp tôi có được nền kiến thức đại cương đủ tốt như ngày hôm nay, để tôi có thể tiếp tục theo đuổi con đường nghiên cứu.
%
%Bên cạnh đó, tôi cũng xin gửi lời cảm ơn đến các thầy trong hội đồng chấm khóa luận tốt nghiệp đã dành thời gian để đọc và góp ý cho tôi những một số lỗi để tôi có thể chỉnh sửa và khóa luận được hoàn chỉnh hơn.
%
%Cuối cùng, tôi xin gửi lời cảm ơn đến quý Thầy Cô phòng Giáo vụ khoa Toán - Tin học và phòng Đào tạo trường Đại học Khoa học Tự nhiên đã tạo điều kiện cho tôi có thể đăng ký học phần Khóa luận Tốt nghiệp này.
%
%Tôi cũng rất mong nhận được những phản hồi, cũng như những lời góp ý, đóng góp chân thành để khóa luận ngày càng hoàn thiện hơn. Mọi liên hệ hoặc thắc mắc xin gửi về địa chỉ sau: \url{nhanmath97@gmail.com}. Tôi xin chân thành cảm ơn.

\begin{flushright}
{\it Tp.HCM, ngày 10 tháng 12 năm 2022}

Tác giả\hskip 2cm \quad \quad

\vskip 2cm

{\bf Phan Thành Nhân} \hskip 1cm \quad\ 
 \end{flushright}
\thispagestyle{empty}

\tableofcontents
\chapter*{Lời nói đầu}
\addcontentsline{toc}{chapter}{Lời nói đầu}

Trong kỷ nguyên của dữ liệu lớn, sự phát triển của dữ liệu đặt ra những thách thức đối với việc quản lý và sử dụng dữ liệu hiệu quả. Ví dụ: một tập dữ liệu gen có thể chứa hàng trăm nghìn đặc điểm \cite{guyon2007competitive}. Do đó, việc xử lý trực tiếp các tập dữ liệu như vậy có thể gặp sự khó khăn về kích thước. Hơn nữa, các đặc trưng dư thừa có thể làm giảm hiệu suất học tập của các thuật toán phân loại. Để giải quyết vấn đề này, nhiều kỹ thuật giảm chiều dữ liệu đã được phát triển \cite{melab2006grid,de2006parallelizing,garcia2006parallel,guillen2009efficient,lopez2006solving} và chúng được phân loại thành các phương pháp trích xuất đặc trưng hoặc lựa chọn đặc trưng \cite{liu2012feature, kumar2014feature}. Kỹ thuật trích xuất đặc trưng (ví dụ: Phân tích thành phần chính (Principal Component Analysis - PCA) \cite{johnson2002applied}, Phân tích phân biệt tuyến tính (Linear Discriminant Analysis - LDA) \cite{johnson2002applied}) liên quan đến việc chiếu dữ liệu vào một không gian đối tượng mới với số chiều nhỏ hơn thông qua một vài bước biến đổi tuyến tính hoặc phi tuyến từ các đặc trưng gốc. Tuy nhiên, điều này tạo ra một loạt các đặc trưng mới mà không thể diễn giải trực tiếp. Hơn nữa, vì những cách tiếp cận đó sử dụng tất cả các đặc trưng có sẵn trong quá trình trích xuất đặc trưng nên nó không giúp giảm chi phí lưu trữ dữ liệu và chi phí thu thập dữ liệu trong tương lai. Mặt khác, các phương pháp lựa chọn đặc trưng (\cite{sinaga2021entropy, james2013introduction},\ldots) chỉ chọn một tập hợp con các đặc trưng hữu ích để xây dựng mô hình. Do đó, điều này giúp giữ các tính chất của các đặc trưng ban đầu trong khi giảm chi phí lưu trữ và thu thập dữ liệu trong tương lai bằng cách loại bỏ các đặc trưng không liên quan. Tuy nhiên, các dữ liệu từ các lĩnh vực khác nhau như khai thác văn bản, phân tích kinh doanh và sinh học, thường được đo bằng gigabyte hoặc terabyte với hàng triệu đặc trưng \cite{bolon2015feature, li2017feature}. Ví dụ: tập dữ liệu Amazon Review \cite{ni2019justifying} là tập dữ liệu 34 gigabyte. Trong những trường hợp như vậy, hiệu suất của các kỹ thuật lựa chọn đặc trưng mới nhất có thể bị ảnh hưởng  \cite{li2017feature}. Điều này là do không gian tìm kiếm cho một tập hợp con các đặc trưng hữu ích bị tăng lên đáng kể. Một cách để giải quyết vấn đề này là sử dụng tính toán song song, cho phép sử dụng tốt hơn tài nguyên tính toán của máy tính bằng cách phân vùng dữ liệu và chạy các lựa chọn đặc trưng trên nhiều lõi cùng một lúc.

Trong đề tài luận văn cao học này, chúng tôi phát triển một phương pháp \lq\lq Sử dụng tính toán song song để lựa chọn đặc trưng dựa trên tiêu chí tỉ lệ vết (Parallel feature selection based on the trace ratio criterion - PFST)\rq\rq cho bài toán phân loại. Tiêu chí được sử dụng là một thước đo về khả năng tách lớp được sử dụng trong LDA, để đánh giá tính hữu dụng của đặc trưng. Dựa trên tiêu chí này, PFST sẽ nhanh chóng tìm thấy các đặc trưng quan trọng từ một tập hợp các đặc trưng gốc của tập dữ liệu lớn bằng cách sử dụng sức mạnh của tính toán song song để thực hiện lựa chọn đặc trưng và loại bỏ các đặc trưng dư thừa. Sau khi các đặc trưng quan trọng nhất được đưa vào mô hình, chúng tôi đánh giá phương pháp thông qua các thử nghiệm khác nhau bằng cách sử dụng LDA làm mô hình phân loại. Thực nghiệm cho thấy rằng phương pháp của chúng tôi có thể chọn ra một tập hợp con nhỏ các đặc trưng trong thời gian ngắn, tiết kiệm thời gian hơn so với một số phương pháp khác. Ngoài ra, độ chính xác khi phân loại dựa trên các đặc trưng được lựa chọn bằng PFST cũng đạt độ chính xác cao và tốt hơn các phương pháp khác, và tốt hơn việc phân loại dựa trên tập các đặc trưng gốc ban đầu.

Nội dung khóa luận này bao gồm 3 chương. Trong đó,
\begin{description}
	\item[Chương 1:] Chương này giới thiệu tổng quan về lựa chọn đặc trưng và các nghiên cứu liên quan.
	\item[Chương 2:] Chương này trình bày về tiêu chí vết - tiếu chí lựa chọn đặc trưng. Thuật toán PSFT - thuật toán lựa chọn đặc trưng song song dựa trên tiêu chí vết.
	\item[Chương 3:] Chương này trình bày về các tập dữ liệu được sử dụng khi làm thực nghiệm. Thông tin về phần cứng được sử dụng và thảo luận kết quả
	\item[Cuối cùng:] Phần kết luận và danh mục các tài liệu tham khảo.
\end{description}

\include{chapter1}
\chapter{Kiến thức nền tảng}

\section{Quy trình của lớp các bài toán phân loại}
Để giải quyết một bài toán phân loại, ta cần phải hiểu dữ liệu, tính chất các đặc trưng và quá trình chọn các đặc trưng phù hợp với mô hình cũng cần phải có một quy trình rõ ràng.

Bước đầu tiên là xác định bài toán, nghĩa là xác định nhãn của bài toán và xây dựng tập dữ liệu (data collection). Phải đảm bảo là tập dữ liệu phải liên quan đến bài toán được mô hình hóa. Bước này rất quan trọng vì sẽ ảnh hưởng rất nhiều đến kết quả phân loại. Ngoài ra, chỉ những đặc trưng có thông tin hữu ích về bài toán là nên được sử dụng. Trong một số trường hợp, khi gặp khó khăn về kiến thức cũng như khả năng thu thập dữ liệu. Chúng ta có thể sử dụng phương pháp brute-force để thay thế. Brute-force là một phương pháp giải quyết vấn đề bằng cách thực hiện tất cả các giải pháp có thể có và chọn ra giải pháp tốt nhất. Nó được sử dụng khi không có một thuật toán cụ thể nào có thể giải quyết vấn đề hoặc khi không có đủ kiến thức về vấn đề để thiết kế một giải pháp tối ưu. Tuy nhiên, phương pháp này có thể rất tốn kém và thời gian, đặc biệt là đối với các vấn đề có kích thước lớn. Do vậy, trong trường hợp này sẽ có một lượng rất lớn các biến được đo, xử lý và thêm vào tập dữ liệu. Tuy nhiên, hi vọng, trong tương lại có thể tác được các đặc trưng tốt nhất và phù hợp nhất với bài toán.

Nếu vấn đề về dữ liệu có thể được giải quyết, chúng ta sẽ nên bước tiếp theo trong quy trình phân loại là tiền xử lý dữ liệu (data pre-processing). Ở bước này, vấn đề chính là thiếu dữ liệu (missing data) và dữ liệu ngoại lai (outlier) cần phải được xử lý. Có một vài phương pháp phân tích thóng kê \cite{aggarwal2001outlier, hodge2004survey} để có thể dử lý các vấn đề này. Hơn nữa, đây là bước mà số lượng các đặc trưng của bài toán có thể giảm đi bằng việc áp dụng thuật toán lựa chọn đặc trưng.

Vấn đề này có thể được giải quyết, đưa chúng ta đến bước tiếp theo trong quy trình phân loại: tiền xử lý dữ liệu. Ở giai đoạn này, các vấn đề chính như giá trị bị thiếu và phát hiện ngoại lệ nên được xử lý. Có nhiều phương pháp phân tích thống kê để giải quyết các vấn đề này [1, 26]. Ngoài ra, đây là giai đoạn trong vấn đề mà số lượng đặc trưng của vấn đề có thể được giảm bằng thuật toán lựa chọn đặc trưng.

Bước tiếp theo là lựa chọn thuật toán phân loại. Có rất nhiều các thuật tóa phân loại, và mặc dù các thuật toán rất đa dạng và khác nhau về ý tưởng, nhưng không dễ dàng chọn được thuật toán nào là tốt nhất cho một bài toán cụ thể. Do đó, việc thử nghiệm và so sánh một số thuật toán là một cách làm tương đối phổ biến, mục tiêu cuối cùng là lựa chọn được thuật toán cho kết quả tốt nhất \cite{kotsiantis2007supervised}.

Việc đánh giá các thuật toán phân loại thường dựa trên độ chính xác của việc dự đoán. Một kỹ thuật điển hình là chia dữ liệu thành hai phần để huấn luyện mô hình và sử dụng phần còn lại để kiểm tra độ chính xác. Tuy nhiên, quy trình này thường dẫn đến kết quả không tốt khi áp dụng vào tập dữ liệu bên ngoài. Do đó, để giảm thiểu sai số, một số kỹ thuật phức tạp hơn như kiểm tra chéo (cross-validation) \cite{kohavi1995study} có thể được sử dụng.

Để có cái nhìn tổng quan, chúng tôi xin trình bày theo sơ đồ khối của toàn bộ quy trình của một bài toán phân loại.

\begin{center}
	\begin{tikzpicture}[thick, scale=0.85, every node/.style={scale=0.9}]
		\path
			(0:0) coordinate (O)
			(0:3) coordinate (A)
			(0:6.5) coordinate (B)
			(0:10) coordinate (C)
			(0:13.5) coordinate (D)
			(0:17) coordinate (E)
			(D) ++ (-90:3) coordinate (F)
			(E) ++ (-90:5) coordinate (G)
			(G) ++ (-90:3) coordinate (H)
		;
		\draw 
			(O) circle (1)
			($(A)+(180:1.25)+(-90:0.5)$) rectangle ($(A)+(0:1.25)+(90:0.5)$)
			($(B)+(180:1.25)+(-90:0.5)$) rectangle ($(B)+(0:1.25)+(90:0.5)$)
			($(C)+(180:1.25)+(-90:1)$) rectangle ($(C)+(0:1.25)+(90:1)$)
			($(D)+(180:1.25)+(-90:0.5)$) rectangle ($(D)+(0:1.25)+(90:0.5)$)
			($(E)+(180:1.25)+(-90:0.5)$) rectangle ($(E)+(0:1.25)+(90:0.5)$)
			
			($(F)+(180:1.25)+(-90:0.5)$) rectangle ($(F)+(0:1.25)+(90:0.5)$)
			
			(G)+(180:1)--+(90:1)--+(0:1)--+(-90:1)--cycle
			
			($(H)+(180:1.25)+(-90:0.5)$) rectangle ($(H)+(0:1.25)+(90:0.5)$)
		;
		\path 
			(O) ++ (90:0.25) node{Xác định} ++ (-90:0.5) node{bài toán}
			
			(A) ++ (90:0.25) node{Xây dựng} ++ (-90:0.5) node{tập dữ liệu}
			
			(B) ++ (90:0.25) node{Tiền xử lý} ++ (-90:0.5) node{dữ liệu}
			
			(C) ++ (90:0.5) node{Lựa chọn} ++ (-90:0.5) node{mô hình} ++ (-90:0.5) node{phân loại}
			
			(D) ++ (90:0.25) node{Huấn luyện} ++ (-90:0.5) node{mô hình}
			
			(E) ++ (90:0.25) node{Đánh giá} ++ (-90:0.5) node{mô hình}
			
			(F) ++ (90:0.25) node{Điều chỉnh} ++ (-90:0.5) node{tham số}
			(G) ++ (90:0.25)node {Kiểm} ++ (-90:0.5)node{tra}
			(H) node {Kết thúc}
		;
		\draw[->]
		($(O) + (0:1)$) -- ($(A)+(180:1.25)$)
		;
		
		\draw[->] ($(A) + (0:1.25)$) -- ($(B)+(180:1.25)$);
		\draw[->] ($(A) + (-90:5)$) -- ($(A)+(-90:0.5)$);
		
		\draw[->] ($(B) + (0:1.25)$) -- ($(C)+(180:1.25)$);
		\draw[->] ($(B) + (-90:5)$) -- ($(B)+(-90:0.5)$);
		
		\draw[->] ($(C) + (0:1.25)$) -- ($(D)+(180:1.25)$);
		\draw[->] ($(C) + (-90:5)$) -- ($(C)+(-90:1)$);
		
		\draw[->] ($(D) + (0:1.25)$) -- ($(E)+(180:1.25)$);
		\draw[->] ($(D) + (-90:5)$) -- ($(F)+(-90:0.5)$);
		\draw[->] ($(F) + (90:0.5)$) -- ($(D)+(-90:0.5)$);
		
		\draw[->] ($(D) + (0:1.25)$) -- ($(E)+(180:1.25)$);
		\draw[->] ($(E) + (-90:0.5)$) -- ($(G)+(90:1)$);
		\draw[->] ($(G) + (-90:1)$) -- node[right]{Đạt} ($(H)+(90:0.5)$);
		
		\draw ($(A) + (-90:5)$) -- ($(D)+(-90:5)$) -- node[above]{Chưa đạt} ($(G)+(180:1)$);
	\end{tikzpicture}
\end{center}

Trong toàn bộ quy trình, nếu có bất kì bước nào không tốt, quy trình phải quy trở lại bước trước đó. Có nhiều nguyên nhân có thể ảnh hưởng đến hiếu suất của một bài toán phân loại \cite{kotsiantis2007supervised} như:
\begin{itemize}
	\item Đặc trưng phù hợp không được lựa chọn tốt.
	\item Tập dữ liệu không đủ, ít mẫu quan sát.
	\item Số lượng các đặc trưng quá nhiều.
	\item Kỹ thuật tiền xử lý dữ liệu chưa được tốt.
	\item Mô hình phân loiaj được chọn không phù hợp cho vấn đề hoặc cần điều chỉnh tham số.
\end{itemize}

Vì vậy, không thể chỉ ra rõ ràng bước nào trong quy trình cần trở lại. Tuy nhiên, mục tiêu cuối cùng là giải quyết bài toán phân loại đạt kết quả tốt nhất cho dữ liệu chưa được quan sát. Đây là bài toán khó, và mỗi bước thường được thực hiện trong thời gian dài, thông thường chúng ta cần phải liên tục thực hiện nhiều thử nghiệm mới để cải thiện khả năng dự đoán của mô hình.

\section{Lựa chọn đặc trưng}
Ngày nay, các phương pháp học máy xuất hiện ngày càng nhiều và rất mạnh mẽ để giải quyết các bài toán dữ liệu lớn. Các mô hình học máy phổ biến hiện nay như cây quyết định (decision tree), rừng ngẫu nhiên (random forest), SVM, KNN,\ldots đều là những mô hình mạnh mẽ, linh hoạt và có độ chính xác cao cả trong bài toán phân loại hay bài toán hồi quy. Tuy nhiên, bên cạnh việc áp dụng các mô hình học máy, chúng ta cần phải chuẩn hóa dữ liệu tốt, bởi vì dữ liệu là nguyên liệu để mô hình học máy học dựa trên đó. Kết quả của một bài toán sử dụng học máy có thể sẽ được cải thiện rõ rệt nếu có bước chuẩn bị dữ liệu tốt. Và việc lựa chọn đặc trưng là một kĩ thuật quan trọng bênh cạnh việc trích xuất đặc trưng hay biến đổi đặc trưng. Trong phạm vi của khóa luận này, chúng tôi chỉ xin nhắc lại khái quát về lựa chọn đặc trưng và các thuật toán liên quan.

\subsection{Lựa chọn đặc trưng là gì?}
Để xây dựng mô hình, chúng ta sẽ cần đến thông tin. Thông tin đến tự những bộ dữ liệu, nhưng với sự bùng nổ của dữ liệu lớn (bigdata), dữ liệu dường như trở nên quá nhiều, khiến việc xây dựng mô hình gặp nhiều khó khăn như tăng chi phí tính toán, quá nhiều đặc trưng có thể dẫn tới hiện tượng quá khớp (overfitting) - là hiện tượng mô hình hoạt động tốt trên tập huấn luyện (training set), nhưng tệ trên tập thử nghiệm (testing set), một số đặc trưng có thể gây nhiễu và làm giảm chất lượng mô hình,\ldots

Có rất nhiều thuật toán lựa chọn đặc trưng đã được phát triển từ rất lâu đến tận thời điểm hiện tại. Trong đó, lựa chọn tiến (forward feature selection), lựa chọn lùi (backward feature selection) và lựa chọn từng bước (stepwise feature selection) là ba thuật toán rất phổ biến. Trong phần này, chúng tôi sẽ tóm tắt lại ba kĩ thuật này, và trình bày chi tiết cách chúng tôi sử dụng trong các thuật toán \ref{alg:ffs}, \ref{alg:bfs}, và \ref{alg:sfs}.

Từ đây, để thuận tiện trong việc trình bày, chúng tôi sẽ sử dụng \lq\lq ffs\rq\rq~ để chỉ thuật toán lựa chọn tiến, \lq\lq bfs\rq\rq~ để chỉ thuật toán lựa chọn lùi và \lq\lq sfs\rq\rq~ để chỉ thuật toán lựa chọn từng bước.

\subsection{Thuật toán lựa chọn tiến}
Ffs được sử dụng rất rộng rãi vì sự hiệu quả trong việc tính toán của nó, cùng với khả năng xử lý hiệu quả các vấn đề bao gồm việc số lượng đặc trưng vượt quá số lượng quan sát. Tuy nhiên, một số đặc trưng có thể xuất hiện dư thừa sau khi đã lựa chọn các đặc trưng khác. Về các điều kiện đủ để lựa chọn tiến nhằm khôi phục mô hình ban đầu và tính ổn định của nó, chúng tôi tham khảo từ \cite{tropp2004greed} và \cite{donoho2005stable}. Dưới đây là chi tiết thuật toán

\begin{breakablealgorithm}
	\caption{\textbf{Lựa chọn tiến}\\
		(Forward Feature Selection)}\label{alg:ffs}
	\noindent\textbf{Input:} Một tập dữ liệu gồm $p$ đặc trưng $f_1,f_2,...,f_p$, tham số tiến là $\alpha$.\\
	\textbf{Output:} Một tập hợp $R$ gồm các đặc trưng được chọn.
	\begin{algorithmic}[1]
		\State $R \gets \emptyset$
		\State $S \gets \{f_1, f_2,\ldots, f_p\}$
		\While {True}
		\State $f_j \gets$ đặc trưng hữu ích nhất trong $S$
		\If{mô hình được cải thiện tốt hơn một lượng là $\alpha$ sau khi thêm vào $f_j$}
		\State $R \gets R \cup \{f_j\}$
		\State $S \gets S\setminus \{f_j\}$
		\Else
		\State \Return $R$
		\EndIf
		\EndWhile
	\end{algorithmic}
\end{breakablealgorithm}

\subsection{Thuật toán lựa chọn lùi}
Bfs \cite{james2013introduction} được trình bày chi tiết trong thuật toán \ref{alg:bfs}. Bắt đầu với toàn bộ đặc trưng, sau đó lần lượt loại bỏ các đặc trưng ít hữu ích nhất từ từ, mỗi lần một đặc trưng. Yêu cầu của bfs bảo đảm những đặc trưng dư thừa được loại bỏ khỏi mô hình. Tuy nhiên, bfs thì có tốc độ tính toán khá chậm, và chậm hơn nhiều khi so với fsf.

\begin{breakablealgorithm}
	\caption{\textbf{Lựa chọn lùi}\\
		(Backward Feature Selection)}\label{alg:bfs}
	\noindent\textbf{Input:} Một tập dữ liệu gồm $p$ đặc trưng $f_1,f_2,...,f_p$, tham số lùi là $\beta$.\\
	\textbf{Output:} Một tập hợp $R$ gồm các đặc trưng được chọn.
	\begin{algorithmic}[1]
		\State $R \gets \{f_1, f_2,\ldots, f_p\}$
		\While {True,}
		\State $f_j \gets$ là đặc trưng ít hữu ích nhất trong R.
		\If {giá trị mất mát của mô hình sau khi loại $f_j$ là nhỏ hơn $\beta$}
		\State $R \gets R\setminus \{f_j\}$
		\Else 
		\State \Return $R$
		\EndIf
		\EndWhile
	\end{algorithmic}
\end{breakablealgorithm} 

\subsection{Thuật toán lựa chọn từng bước}
Ngoài hai thuật toán được nêu ra ở trên, một thuật toán khác, là phiên bản kết hợp của cả thuật toán lựa chọn tiến và thuật toán lựa chọn lùi là thuật toán lựa chọn từng bước, được trình bày chi tiết trong thuật toán \ref{alg:sfs}. Trong cách tiếp cận này, các đặc trưng được thêm vào mô hình một cách có tuần tự như trong lựa chọn tiến. Tuy nhiên, sau khi thêm vào các đặc trưng mới, phương pháp này có thể loại bỏ bất kỳ đặc trưng nào mà có vẻ không còn phù hợp.

\begin{breakablealgorithm}
	\caption{\textbf{Lựa chọn từng bước}\\
		(Stepwise Feature Selection)}\label{alg:sfs}
	%\hspace*{\algorithmicindent} 
	\noindent\textbf{Input:} Một tập hợp gồm $p$ đặc trưng $f_1,f_2,...,f_p$, tham số tiến là $\alpha$, tham số lùi là $\beta$.\\
	\textbf{Output:} Một tập hợp $R$ gồm các đặc trưng được chọn.
	\begin{algorithmic}[1]
		\State $R \gets \emptyset$
		\State $S \gets \{f_1, f_2,\ldots, f_p\}$
		\While {True,}
		\State $f_j \gets$ là đặc trưng hữu ích nhất trong $S$
		\If{mô hình cải thiện hơn một lượng là $\alpha$ sau khi thêm $f_j$}
		\State $R \gets R \cup \{f_j\}$
		\State $S \gets S\setminus \{f_j\}$
		\While{True}
		\State {$f_k\gets$ là đặc trưng ít hữu ích nhất trong R}
		\If {hiệu suất của mô hình giảm một lượng nhỏ hơn $\beta$ sau khi loại bỏ $f_j$.}
		\State $R \gets R\setminus \{f_k\}$
		\Else {\;break}
		\EndIf
		\EndWhile\;
		\Else 
		\State{\;\Return $R$}
		\EndIf
		\EndWhile
	\end{algorithmic}
\end{breakablealgorithm}

\section{Các nghiên cứu liên quan}
Đối mặt với các thách thức về kích thước của dữ liệu ngày càng tăng, đã có nhiều nỗ lực trong hướng nghiên cứu về lựa chọn đặc trưng để phát triển các kỹ thuật mới. Bên cạnh các phương pháp lai để kết hợp các chiến lược lựa chọn đặc trưng khác nhau \cite{saeys2007review,ang2015supervised}, hầu hết các phương pháp lựa chọn đặc trưng có thể được chia thành ba loại.

Đầu tiên, cách tiếp cận \lq\lq wrapper\rq\rq~ dựa trên hiệu suât của một thuật toán học máy cụ thể để đánh giá tầm quan trọng của các đặc trưng được chọn. Một phương pháp \lq\lq wrapper\rq\rq~ điển hình sẽ tìm kiếm một tập con các đặc trưng dựa trên một thuật toán học máy trước, sau đó sẽ đánh giá chúng. Các bước này được lặp lại cho đến khi thỏa mãn một số tiêu chí dừng. Các phương pháp trong loại này thường rất tốn kém về chi phí tính toán vì việc đánh giá tập con các đặc trưng yêu cầu nhiều lần lặp lại. Mặc dùng nhiều các tiếp cận tìm kiếm được đề xuất chẳng hạn như thuật toán tìm kiếm best-first \cite{arai2016unsupervised} và thuật toán di chuyền (genetic) \cite{goldberg1988genetic}. Tuy nhiên, việc sử dụng các thuật toán này cho dữ liệu nhiều chiều vẫn không cho thấy sử cải thiện về chi phí tính toán.

Thứ hai, cách tiếp cận \lq\lq filter\rq\rq~ bao gồm các kỹ thuật đánh giá các tập hợp con đặc bằng việc xếp hạng với một số tiêu chí như tiêu chí thông tin \cite{nguyen2014efficiency, shishkin2016efficiency}, khả năng tái tạo \cite{farahat2011efficiency, masaeli2010convex}. Các phương pháp này chọn các đặc trưng độc lập với thuật toán học máy và thường hiệu quả hơn về chi phí tính toán so với các phương pháp \lq\lq wrapper\rq\rq~\cite{li2017feature}. Tuy nhiên, vì không được tối ưu hóa cho bất kỳ thuật toán học máy mục tiêu nào, nên chúng có thể không tối ưu cho một tuật toán học máy cụ thể.

Thứ ba, các phương pháp \lq\lq embedded\rq\rq~ sử dụng các tiêu chí độc lập để tìm ra tập con tối ưu cho một tập hợp nhất định. Sau đó, một thuật toán học máy được sử dụng để lựa chọn tập con tối ưu cuối cùng trong số các tập con tối ưu trên các tập hợp khác nhau. Vì thế, chúng hiệu quả hơn về chi phí tính toán so với các phương pháp \lq\lq wrapper\rq\rq~ vì chúng không đánh giá đặc trưng dựa trên việc lặp lại các tập con đặc trưng. Ngoài ra, chúng cũng được huấn luyện từ các thuật toán học máy. Vì thế, chúng có thể được coi như sự đánh đổi giữa phương pháp \lq\lq filter\rq\rq~ và phương pháp \lq\lq wrapper\rq\rq \cite{li2017feature}.

Mặc dù, cho đến này, các nhà khoa học đã nỗ lực rất nhiều trong hướng nghiên cứu lựa chọn đặc trưng, nhưng dữ liệu từ các trường, các ngành khác nhau có thể quá phong phú ngay cả đối với các phương pháp \lq\lq filter\rq\rq hiệu quả về chi phí tính toán. Điều này đã thúc đẩy nhiều nghiên cứu khác trong việc lựa chọn đặc trưng song song. Một số phương pháp đã được đề xuất trong \cite{melab2006grid,de2006parallelizing,garcia2006parallel,guillen2009efficiency,lopez2006solve} sử dụng quy trình xử lý song song để đánh giá đồng thời nhiều đặc trưng. Tuy nhiên, các thuật toán này yêu cầu quyên truy cập vào toàn bộ dữ liệu. Mặc khác, trong trong \cite{singh2009parallel}, các tác giả đã đề xuất một thuật toán lựa chọn đặc trưng song song cho hồi quy logistic dựa trên framework MapReduce và các đặc trưng được đánh giá thông qua hàm mục tiêu của mô hình hồi quy logistic. Trong khi đó, các tác giả của bài báo \cite{tsamardinos2019greedy} đã đề xuất \textit{Song song, Tiến–Lùi với thuật toán Tỉa (Parallel, Forward–Backward with Pruning algorithm)} (PFBP) để lựa chọn đặc trưng bằng cách bỏ sớm một số đặc trưng trong các lần lặp lại tiếp theo và sớm trả ra kết quả đặc trưng tốt nhất trong mỗi lần lặp. Tuy nhiên, các tiếp cận này yêu cầu tính toán bootstrap của p-giá trị, rất tốn kém chi phí tính toán. Trong bài báo \cite{zhao2013massively}, Zhao và các cộng sự đã giới thiệu một thuật toán lựa chọn đặc trưng song song để chọn các đặc trưng dựa trên khả năng giải thích phương sai của dữ liệu. Tuy nhiên, theo các tiếp cận của họ, việc xác định số lượng các đặc trưng trong mô hình dựa trên việc chuyển đổi các nhãn phân loại thành các giá trị số và sử dụng tổng bình phương sai số. Việc tính tổng bình phương sai số đòi hỏi phải điều chỉnh mô hình và do đó thuật toán vẫn còn tốn kém rất nhiều chi phí tính toán.
\include{chapter3}
\chapter{Thực nghiệm}\label{exper}
\begin{table}[htbp]
	\caption{Các tập dữ liệu được sử dụng trong thực nghiệm}
	\begin{center}
		\begin{tabular}{|c|c|c|c|}
			\hline
			Tập dữ liệu & \# Số lớp & \# Số đặc trưng & \# Kích thước mẫu \\
			\hline
			Breast cancer & $2$ & $30$ & $569$ \\
			\hline
			Parkinson & $2$ & $754$ & $756$ \\
			\hline
			Mutants & $2$  & $5408$ & $31419$\\
			\hline
			Gene & $5$ & $20531$ & $801$ \\
			\hline
			Micromass & $10$ & $1087$ & $360$\\
			\hline
		\end{tabular}
		\label{table_info_datasets}
	\end{center}
\end{table}

Để mô tả hiệu suất của PFST, chúng tôi so sánh PFST với một số kĩ thuật gồm
\begin{itemize}
	\item Parallel Sequential Forward Selection (PSFS) \cite{scikit-learn},
	\item Parallel Sequential Backward Selection (PSBS) \cite{scikit-learn},
	\item Parallel Support Vector Machine Feature Selection based on Recursive Feature Elimination with Cross-Validation (PSVMR) \cite{guyon2002gene},
	\item Parallel Mutual Information-based Feature Selection (PMI) \cite{bennasar2015feature}.
\end{itemize}

\section{Thông tin về các tập dữ liệu và các thiết lập}
Các thực nghiệm đã được hoàn thành trên các tập dữ liệu lấy từ thư viện Scikit-learn \cite{scikit-learn} và kho dữ liệu máy học UCI \cite{Dua:2019}. Thông tin số liệu của các tập dữ liệu này được trình bày trong bảng \ref{table_info_datasets}.

Chúng tôi đã cộng một lượng nhỏ nhiễu vào tập dữ liệu Gene và tập dữ liệu Mutants để tránh lỗi không tìm được nghịch đảo của các ma trận đơn khi chạy thuật toán PSFT. Ngoài ra, đối với tập dữ liệu Mutants, chúng tôi đã loại bỏ một dòng mà tất cả các giá trị trong dòng đều là rỗng. 

Với thuật toán PSFT, các tham số được sử dụng là $\alpha=\gamma=0.05$, $\beta=0.01$. Với thuật toán PSBS và PSFS, chúng tôi đã sử dụng thuật toán KNN với $K=3$ làm hàm ước lượng. Với thuật toán PSVMR, chúng tôi sử dụng kernel tuyến tính và sử dụng kernel \lq\lq JMI\rq\rq~ cho thuật toán PMI. Để cho việc so sánh được công bằng, chúng tôi đã set số lượng đặc trưng cần được chọn từ các kĩ thuật khác bằng với số lượng đặc trưng mà PSFT chọn được.

Về cấu hình, chúng tôi chạy thực nghiệm trên một CPU là AMD Ryzen 7 3700X với 8 nhân và 16 luồng, 3.6GHz và 16GB ram. Sau khi chọn được các đặc trưng, chúng tôi thực hiện bài toán phân loại bằng việc sử dụng mô hình phân tích biệt thức tuyến tính (linear discriminant analysis - LDA) và trình bày kết quả của 5-fold misclassification rate trong bảng \ref{table_error}. Ngoài ra, chúng tôi cũng trình bày thời gian chạy và số lượng đặc trưng được chọn từ tất cả đặc trưng ban đầu trong bảng \ref{tab_time}

Chúng tôi sẽ bỏ những trường hợp nếu không nhận được kết quả sau 5 giờ hoặc khi có vấn đề về việc tràn ram, và ký hiueej NA trong bảng \ref{table_error} và \ref{tab_time}) để chỉ những trường hợp như vậy.

\section{Kết quả và thảo luận}

\begin{table*}[htbp]
	\caption{5-fold misclassification rate}
	\resizebox{\textwidth}{!}{%
		\begin{tabular}{|l|c|c|c|c|c|c|c|}
			\hline
			\textbf{Datasets} &
			\textbf{\# Selected Features} & 
			{\textbf{PFST (our)}} &
			{\textbf{PSFS}} &
			{\textbf{PSBS}} &
			{\textbf{PSVMR}} &
			{\textbf{PMI}} &
			\textbf{Full Features} \\ \hline
			\textbf{Breast cancer} &
			$3$ &
			{$\boldsymbol{0.042}$} &
			{$0.111$} &
			{$0.074$} &
			{$0.051$} &
			{$0.076$} &
			$0.042$ \\ \hline
			\textbf{Parkinson} &
			$11$ &
			{$\boldsymbol{0.112}$} &
			{$0.234$} &
			{NA} &
			{NA} &
			{$0.181$} &
			$0.362$ \\ \hline
			\textbf{Mutants} &
			$6$ &
			{\textbf{0.008}} &
			{NA} &
			{NA} &
			{NA} &
			{NA} &
			0.010 \\ \hline
			\textbf{Gene} &
			$12$ &
			{$\boldsymbol{0.006}$} &
			{$0.009$} &
			{NA} &
			{NA} &
			{$0.007$} &
			$0.042$ \\ \hline
			\textbf{Micromass} &
			$19$ &
			{$0.115$} &
			{$0.310$} &
			{$0.218$} &
			{$\boldsymbol{0.096}$} &
			{$0.228$} &
			$0.129$ \\ \hline
		\end{tabular}
		\label{table_error}
	}
\end{table*}

\begin{table*}[htbp]
	\caption{Running time and number of selected features}
	\resizebox{\textwidth}{!}{%
		\begin{tabular}{|l|c|c|ccccc|}
			\hline
			\multicolumn{1}{|c|}{\multirow{2}{*}{\textbf{Datasets}}} &
			\multirow{2}{*}{\textbf{\begin{tabular}[c]{@{}c@{}}\# selected\\ features\end{tabular}}} & \multicolumn{1}{c|}{\multirow{2}{*}{\textbf{\# Features}}} &
			\multicolumn{5}{c|}{\textbf{Running Time (s)}}  \\ \cline{4-8} 
			\multicolumn{1}{|c|}{} &
			& &
			\multicolumn{1}{c|}{\textbf{PFST (our)}} &
			\multicolumn{1}{c|}{\textbf{PSFS}} &
			\multicolumn{1}{c|}{\textbf{PSBS}} &
			\multicolumn{1}{c|}{\textbf{PSVMR}} &
			\multicolumn{1}{c|}{\textbf{PMI}} 
			\\ \hline
			\textbf{Breast cancer} &
			3 & 30& 
			\multicolumn{1}{c|}{\textbf{0.095}} &
			\multicolumn{1}{c|}{1.403} &
			\multicolumn{1}{c|}{8.268} &
			\multicolumn{1}{c|}{12.470} &
			\multicolumn{1}{c|}{0.646}    \\ \hline
			\textbf{Parkinson} &
			11 & 754&
			\multicolumn{1}{c|}{\textbf{3.219}} &
			\multicolumn{1}{c|}{163.32} &
			\multicolumn{1}{c|}{NA} &
			\multicolumn{1}{c|}{NA} &
			\multicolumn{1}{c|}{77.269}   \\ \hline
			\textbf{Mutants} &
			6 & 5408 &
			\multicolumn{1}{c|}{\textbf{674.702}} &
			\multicolumn{1}{c|}{NA} &
			\multicolumn{1}{c|}{NA} &
			\multicolumn{1}{c|}{NA} &
			\multicolumn{1}{c|}{NA} \\ \hline
			\textbf{Gene} &
			12 & 20531 & 
			\multicolumn{1}{c|}{\textbf{172.386}} &
			\multicolumn{1}{c|}{5350.14} &
			\multicolumn{1}{c|}{NA} &
			\multicolumn{1}{c|}{NA} &
			\multicolumn{1}{c|}{2706.45}   \\ \hline
			\textbf{Micromass} &
			19 & 1087&
			\multicolumn{1}{c|}{\textbf{16.1}} &
			\multicolumn{1}{c|}{252.9} &
			\multicolumn{1}{c|}{10561.3} &
			\multicolumn{1}{c|}{46.909} &
			\multicolumn{1}{c|}{144.684} 
			\\ \hline
		\end{tabular}
		\label{tab_time}
	}
\end{table*}  

Từ bảng \ref{table_error} và \ref{tab_time}, ta có thể thấy rằng phương pháp lựa chọn đặc trưng PFST không chỉ có hiệu suất tốt về mặt tốc độ mà con đạt độ chính xác cao cho bài toán phân loại, Ví dụ, với tập dữ liệu \lq\lq Breast cancer\rq\rq, PFST đã chọn ra tập hợp gồm 3 đặc trưng từ 30 đặc trưng ban đầu với chỉ 0.095 giây, trong khi PSBS cần 8.268 giây và PSVMR cần 12.470 giây. Không chỉ có thời gian chạy tốt nhất, phương pháp PFST còn đạt được kết quả phân loại tốt nhất với tỉ lệ phân loại sai chỉ 0.042, đây cũng là tỉ lệ phân loại sai khi sử dụng tất cả đặc trưng. Rõ ràng, kết quả này đã cho thấy, PFST lựa chọn được các đặc trưng tốt, ảnh hưởng thật sự đến kết quả phân loại. Ngoài ra, ta cũng thấy rằng hiệu suất của PFST tốt hơn PSBS - một phiên bản lựa chọn đặc trưng lùi.


Với tập dữ liệu Parkinson, PFST đã lựa chọn ra 11 trong tổng số 754 đặc trưng. Kết quả thu được là tỉ lệ phân loại sai vẫn là thấp nhất với chỉ 0.112, khoảng 33\% tỉ lệ phân loại sai khi sử dụng toàn bộ đặc trưng (0.362), và 66.87\% tỉ lệ phân loại sai của phương pháp tốt thứ hai là PMI (0.181). Điều này cho thấy PFST đã loại bỏ rất hiệu quả các đặc trưng dư thừa và đẩy được hiệu suất lên đáng kể. Về mặt thời gian chạy, PFST chỉ mất 3.219 giây để chọn 11 đặc trưng từ 754 đặc trưng, trong khi PMI cần đến 77.269 giây, PSFS cần 163.32 giây. PSBS và PSVMR không thể thu được kết quả khi thời gian chạy vượt quá 5 giờ.

Trong tập dữ liệu nhiều đặc trưng nhất, tập dữ liệu Gene, PFST đã rút gọn 20531 đặc trưng xuống còn 12 đặc trưng trong chưa đầy 3 phút và đạt được tỉ lệ phân loại sai tốt nhất với chỉ 0.006. Mặc dùng PSFS và PMI cũng đạt được kết quả phân loại tốt với tỉ lệ phân loại sai lần lượt là 0.009 và 0.007, nhưng hai phương pháp này chạy lâu hơn PFST, còn đối với PSBS và PSVMR đã không thể thu được kết quả trong vòng 5 giờ chạy.

Mặc dù PFST là thuật toán nhanh nhất trong số các thuật toán được sử dụng trong phần thực nghiệm này, nhưng phương pháp này cũng làm tốt hơn các phương pháp khác về tỉ lệ phân loại sai với bốn trong năm tập dữ liệu. Với tập dữ liệu Micromass, hiệu suất tốt nhất về kết quả phân loại thuộc về phương pháp PSVMR, và tốt thứ hai là PFST. Tuy nhiên, PSVMR chỉ tốt hơn PFST rất ít, không đáng kể (chỉ thấp hơn 1.9\% tỉ lệ phân loại sai), nhưng thời gian chạy lại gần gấp 3 lần PSFT (46.909 giây so với 16.1 giây).



\chapter*{Kết luận}
\addcontentsline{toc}{chapter}{Kết luận}
Qua khóa luận này, chúng tôi đã trình bày phương pháp PFST, một cách tiếp cận mới về việc lựa chọn song song các đặc trưng từ các tập dữ liệu lớn trong bài toán phân loại. Phương pháp của chúng tôi đã sử dụng tiêu chí vết, là tiêu chí có các thuộc tính để đánh giá các đặc trưng tốt. Chúng tôi ước lượng các tiếp cận này thông qua nhiều thực nghiệm và các tập dữ liệu khác nhau. Chúng tôi sử dụng LDA làm mô hình phân loại trên các đặc trưng được chọn. Thực nghiệm cho thấy phương pháp của chúng tôi có thể chọn ra các đặc trưng tốt với thời gian ngắn hơn khi so với các phương pháp tiếp cận khác. Hơn nữa, phân loại dựa trên các đặc trưng chọn được bằng PFST cũng thu được độ chính xác tốt hơn các phương pháp khác (tốt hơn bốn trong năm tập dữ liệu) và tốt hơn khi chạy mô hình phân loại với toàn bộ đặc trưng. Tuy nhiên, một trong những nhược điểm của PFST là tiêu chí vết chỉ có thể sử dụng cho các đặc trưng liên tục. Do đó, trong tương lại, chúng tôi vẫn sẽ nỗ lực làm việc, và hi vọng sẽ khám phá ra cách mở rộng cho các trường hợp còn lại trong bài toán phân loại.

Ngoài ra, chúng tôi cũng đã xây dựng thuật toán như một thư viện trong python. Người dùng có thể tìm thấy và cài đặt trong github sau: \url{https://github.com/pthnhan/PFST}
%\addcontentsline{toc}{chapter}{Tài liệu tham khảo}
\begin{thebibliography}{9}
	\bibitem{Ajtai96}M. Ajtai. \textit{Generating hard instances of lattice problems. In Proc. 28th ACM Symp. on Theory of Computing, pages 99–108, 1996. Available from ECCC at http://www.uni-trier.de/eccc/.}
	\bibitem{main} Daniele Micciancio, Oded Regev, \textit{Worst-case to Average-case Reduction based on Gaussian Measures} 
	\bibitem{Peikert1} Chris Peikert, \textit{Latiices in cryptography - Lecture 1 - Mathematical Background}.
	\bibitem{Regev9} Oded Regev, \textit{Lattices in Computer Science - Lecture 9 - Fourier transform}
	\bibitem{MiGo} D. Micciancio and S. Goldwasser, \textit{Complexity of Lattice Problem: A Cryptographic Perspective, volume 671 of The Kluwer International Series in Engineering and Computer Science}. Kluwer Academic Publishers, Boston, Massachusetts, Mar. 2002
	\bibitem{Baibai86} L. Baibai. \textit{On Lovasz' lattice reduction and the nearest lattice point problem. Combinatorica, 6(1):1–13, 1986. Preliminary version in STACS 1985.}
	\bibitem{Mic02} D. Micciancio. \textit{Generalized compact knapsacks, cyclic lattices, and efficient one-way functions from worst-case complexity assumptions. Technical Report TR04-095, ECCC Electronic Colloquium on Computational Complexity, 2004. Preliminary version in FOCS 2002}.
	\bibitem{Ban93}W. Banaszczyk. \textit{New bounds in some transference theorems in the geometry of numbers. Mathematische Annalen, 296(4):625–635, 1993}
	\bibitem{YCai03}J.-Y. Cai. \textit{A new transference theorem in the geometry of numbers and new bounds for Ajtai’s connection factor. Discrete Applied Mathematics, 126(1):9–31, Mar. 2003. Preliminary version in CCC 1999.}
	\bibitem{Regev03} O. Regev. \textit{New lattice-based cryptographic constructions. Journal of the ACM, 51(6):899–942, 2004. Preliminary version in STOC 2003}.
	\bibitem{AhaReg04} D. Aharonov and O. Regev. \textit{Lattice problems in NP intersect coNP. Journal of the ACM, 52(5):749–765, 2005. Preliminary version in FOCS 2004}.
	\bibitem{Mic02-STOC} D. Macciancio. \textit{Almost perfect lattices, the covering radius problem, and applications to Ajtai’s connection factor. SIAM Journal on Computing, 34(1):118–169, 2004. Preliminary version in STOC 2002}.
	\bibitem{HaPham02} Hà Huy Khoái - Phạm Huy Điển. \textit{Số học thuật toán, NXB ĐHQG Hà Nội, 2002}.
\end{thebibliography}
%\newpage
%\addcontentsline{toc}{chapter}{\indexname}
%\printindex
%\bibliographystyle{unsrt}
%\bibliography{bbli}
%\printbibliography[heading=none]
\bibliographystyle{plain}
\bibliography{bbli}
%\chapter*{Phụ lục}
\addcontentsline{toc}{chapter}{Phụ lục}

Trong khóa luận này, chúng tôi đã trình bày về thuật toán PFST, một thuật toán lựa chọn đặc trưng song song dựa trên tiêu chí vết và tối ưu được tốc độ tính toán cũng như hiệu suất của mô hình cho bài toán phân loại. Trong phần này, chúc tôi xin gửi các đoạn mã được viết bằng python cho từng giai đoạn của thuật toán, để người đọc có thể hiểu được thuật toán rõ hơn. Ngoài ra, chúng tôi cũng đã xây dựng thuật toán như một thư viện trong python. Người dùng có thể tìm thấy và cài đặt trong github sau \url{https://github.com/pthnhan/PFST}

\section*{Tiêu chí tỉ lệ vết}
\begin{lstlisting}[language=Python]
import numpy as np
from numpy.linalg import inv

# return trace ratio for univariate case
def univariate(x,y,ni, C):
x = x.flatten()
xbar = np.mean(x)
num_i = lambda i: sum(y==i)*(np.mean(x[y==i])-xbar)**2
numerator = sum(np.array([num_i(i) for i in np.arange(C)]))
den_i = lambda i: (ni[i]-1)*np.var(x[y==i])
denominator = sum(np.array([den_i(i) for i in np.arange(C)]))
return numerator/denominator #np.divide(class_mean_diff, denominator)

# compute trace ratio
def mult_trace(Xmat,y,ni,g):
xbar = np.mean(Xmat, axis = 0)
Sw = 0
Sb = 0
for cl in np.arange(g):
u = np.mean(Xmat[y==cl,:], axis=0)-xbar
Sb += ni[cl]*np.outer(u,u)
Sw += (ni[cl]-1)*np.cov(Xmat[y==cl,:].T)
return np.trace(np.matmul(inv(Sw),Sb))
\end{lstlisting}

\end{document}
